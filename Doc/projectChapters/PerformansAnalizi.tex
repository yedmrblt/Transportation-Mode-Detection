\chapter{Performans Analizi}

Yapılan deneylerin sonuçlarından elde edilen analiz aşağıdaki gibidir:
\begin{itemize}
    \item Tablo 7.1'de belirtilen 1. güzergaha ait Şekil 7.1'deki uygulama ekran çıktısı incelendiğinde sistem hafif raylı sınıfını metro-marmaray sınıfı ile karıştırmaktadır. Yürüme, tramvay ve araba sınıfları başarılı bir şekilde ayırt edilmektedir.
    \item Tablo 7.2'de belirtilen 2. güzergaha ait Şekil 7.2'deki uygulama ekran çıktısı incelendiğinde sistem otobüs-metrobüs sınıfını hafif raylı sınıfı ile karıştırmaktadır. Yürüme, metro-marmaray sınıfları başarılı bir şekilde ayırt edilmektedir.
    \item Yürüme sınıfı en düşük ivmelenmeye sahip olduğu için sistem bu sınıfı diğer sınıflardan kolayca ayırt edebilmektedir.
    \item Kullanıcı eğer çok yavaş yürürse sistem bu sınıfı ayırt edememektedir ve metro-marmaray sınıfı ile karıştırmaktadır.
    \item Hafif raylı vasıtasının ortalama hızı metro ve marmaray vasıtalarının ortalama hızına yakın olması sebebiyle sistem hafif raylı ile metro-marmaray sınıflarını birbirine karıştırmaktadır.
    \item Tramvay sınıfının hızı diğer raylı vasıtalardan daha az olması sebebiyle sistem tarafından bu sınıf kolayca ayırt edilebilmektedir.
    \item Genel olarak sistem 6 sınıf (araba, yürüme, metro-marmaray, tramvay, hafif raylı, otobüs-metrobüs) içerisinden araba, yürüme, metro-marmaray ve tramvay sınıflarını başarılı bir şekilde ayırt edebilmektedir.
\end{itemize}

